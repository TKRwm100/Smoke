\documentclass[paper={550pt,1200pt},lualatex , ja=standard]{bxjsreport}
\usepackage{luatexja-otf}
\usepackage[deluxe]{luatexja-preset}
\usepackage{url}
\usepackage{listings}
\usepackage{hyperref}
\usepackage{xcolor}
\usepackage{tcolorbox}
\usepackage{tikz}
\usepackage{bxpapersize}
\usepackage{layout}
\usepackage{geometry}
\usepackage{siunitx}

\setlength{\hoffset}{-1in+50pt}
\setlength{\voffset}{-1in+50pt}
\setlength{\textwidth}{450pt}
\setlength{\topmargin}{0pt}
\setlength{\headheight}{0pt}
\setlength{\headsep}{0pt}
\setlength{\textheight}{1100pt}
\tcbuselibrary{listings}
\hypersetup{%
luatex,
pdfencoding=auto,
colorlinks=true,%
allcolors=blue
}

\definecolor{White}{rgb}{1,1,1}
\definecolor{Orange}{rgb}{0.9,0.1,0}
\lstset{
  basicstyle={\ttfamily\scriptsize}, % 使用フォント
  classoffset=1,
  breaklines=true,
  identifierstyle={},
  commentstyle={},
  keywordstyle={\bfseries},
  ndkeywordstyle={},
  stringstyle={\ttfamily},
  columns=fixed,
  basewidth=0.5em,
  numberstyle={\tiny},
  stepnumber=1,
  tabsize=4,
  keywordstyle={\color{blue}}, %キーワード(int, ifなど)の書体指定
  commentstyle={\color{OliveGreen}}, %注釈の書体
  stringstyle={\color{Orange}}, %文字列
  showstringspaces=false,  %文字列中の半角スペースを表示させない
  keepspaces=true,
  rulesep = 1pt,
  xrightmargin=0zw,                     % 右マージンのサイズ.
  xleftmargin=1.6zw,                    % 左マージンのサイズ.行番号が2桁でも行左端からはみ出ない値.
}

\makeatletter
\def\lst@lettertrue{\let\lst@ifletter\iffalse}
\makeatother

\pagestyle{empty}
\begin{document}

\def\xmlcol{-20pt}
\def\xmlrowVal{0pt}
\def\xmlrowValTL{10pt}
\def\xmlrowDef{150pt}
\def\xmlrowExp{200pt}

\definecolor{backcolor}{rgb}{0.9,0.9,0.9}
\definecolor{framecolor}{rgb}{0.8,0.8,0.8}
\definecolor{xsdbackcolor}{rgb}{0.9,0.9,1}
\definecolor{xsdframecolor}{rgb}{0.7,0.7,0.9}
\definecolor{White}{rgb}{1,1,1}

\newcommand{\mktree}[3]{
    \node[anchor=west,font=\fontsize{\xmlrowValTL}{\xmlrowValTL}\selectfont] at(\xmlrowVal+\xmlrowValTL*#2,\xmlcol*#3){#1};
    \ifnum #2=0
    \else
    \draw (\xmlrowVal+\xmlrowValTL*#2-\xmlrowValTL/4,\xmlcol*#3)--(\xmlrowVal+\xmlrowValTL*#2+\xmlrowValTL/4,\xmlcol*#3);
    \fi
    }
\newcommand{\drawtreeline}[3]{
    \draw (\xmlrowVal+\xmlrowValTL*#1-\xmlrowValTL/4,\xmlcol*#2+\xmlcol/2)--(\xmlrowVal+\xmlrowValTL*#1-\xmlrowValTL/4,\xmlcol*#3);
    }
\def\postbreak{}
%\def\postbreak{\mbox{\textcolor{red}{$\hookrightarrow$}\space}}
\newtcblisting{reflisting}[2][]{
      arc=5pt,
      top=10pt,
      bottom=10pt,
      left=0pt,
      right=0pt,
      boxrule=1pt,
      colback=backcolor,
      colframe=framecolor,
      listing only,
      hbox,
      #1,
      listing options={
        breaklines=true,
        postbreak=\postbreak,
        #2
      }
}
\newtcblisting{refxsdlisting}[1][]{
      arc=5pt,
      top=10pt,
      bottom=10pt,
      left=15pt,
      right=0pt,
      boxrule=1pt,
      colback=xsdbackcolor,
      colframe=xsdframecolor,
      listing only,
      hbox,
      #1,
      coltitle=black,
      listing options={
        language=XML,
        breaklines=true,
        postbreak=\postbreak,
        escapechar=!,
      }
}

%\layout
\chapter*{Smoke}

\section*{設定ファイル解説}
\begin{refxsdlisting}[title=XSD定義,]
<?xml version="1.0" encoding="utf-8"?>
<xs:element name ="Config">
	<xs:complexType>
		<xs:sequence>
			<xs:element name="Drain">
				<xs:complexType>
					<xs:sequence>
						<xs:element name="Structure" type="Structure" minOccurs="0" maxOccurs="unbounded"/> 
					</xs:sequence>
					<xs:attribute name="X" type="xs:double"/>
					<xs:attribute name="Y" type="xs:double"/>
					<xs:attribute name="Z" type="xs:double"/>
					<xs:attribute name="GenerateSpan" type="xs:time"/>	
					<xs:attribute name="LifeSpan" type="xs:time"/>	
                    <xs:attribute name="ContinueTime" type="xs:time"/>				
				</xs:complexType>
			</xs:element>
			<xs:element name="Chimney">
				<xs:complexType>
					<xs:sequence>
						<xs:element name="Structure" type="Structure" minOccurs="0" maxOccurs="unbounded"/> 
					</xs:sequence>
					<xs:attribute name="X" type="xs:double"/>
					<xs:attribute name="Y" type="xs:double"/>
					<xs:attribute name="Z" type="xs:double"/>
					<xs:attribute name="GenerateSpan" type="xs:time"/>
					<xs:attribute name="LifeSpan" type="xs:time"/>
				</xs:complexType>
			</xs:element>
		</xs:sequence>
	</xs:complexType>
</xs:element>
\end{refxsdlisting}
(要   素) ルート要素です. それぞれ1個の Drain要素, Chimney要素 を持ちます.
\subsubsection*{Drain}
(子 要 素) ドレンの煙描画の設定をします. 0個以上のStructure要素, X, Y, Z, GeneralSpan, LifeSpan,属性を持ちます. 詳しくは下記それぞれの項目を確認してください.
\subsubsection*{Chimney}
(子 要 素) ドレンの煙描画の設定をします. 0個以上のStructure要素, X, Y, Z, GeneralSpan, LifeSpan,属性を持ちます. 詳しくは下記それぞれの項目を確認してください.
\subsubsection*{X}
(子要素属性) ストラクチャの生成場所のX座標を記述します. 自列車原点からの相対座標である必要があります.
\subsubsection*{Y}
(子要素属性) ストラクチャの生成場所のY座標を記述します. 自列車原点からの相対座標である必要があります.
\subsubsection*{Z}
(子要素属性) ストラクチャの生成場所のZ座標を記述します. 自列車原点からの相対座標である必要があります.
\subsubsection*{GenerateSpan}
(子要素属性) ストラクチャの生成周期を記述します.
\subsubsection*{LifeSpan}
(子要素属性) ストラクチャの寿命を記述します.

\subsection*{Structure}
\begin{refxsdlisting}[title=XSD定義,]
<xs:complexType name="Structure">
    <xs:attribute name="Path" type="xs:anyURI"/>
</xs:complexType>
\end{refxsdlisting}
(要   素) ストラクチャ情報を設定します. 0個の子要素を持ちます.
\subsubsection*{Path}
(属   性) 各ストラクチャのモデルのファイルを指定します.
\end{document}